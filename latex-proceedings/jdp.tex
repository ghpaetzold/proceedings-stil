%%%%%%%%%%%%%%%%%%%%%%%%%%%%%%%%%%%%%%%%%%%%%%%%%%%%%%%%%%%%%%%%%%%%%%%%%%%%%%%%%%%%%%%%%%%%%%%%%%%%%%%%%%%%%%%%%%%%%

%Preface:

\section*{Preface}

Evento sat\'elite do XI Brazilian Symposium in Information and Human
Language Technology (STIL 2017), em Uberl\^andia.

A Jornada de Descri\c{c}\~ao do Portugu\^es (JDP), mais uma vez, visa
aproximar as comunidades de linguistas e de pesquisadores da \'area da
Computa\c{c}\~ao. A inten\c{c}\~ao \'e integrar, ainda mais efetivamente, essas duas
\'areas que, especialmente no \^ambito brasileiro, precisam refor\c{c}ar a
atua\c{c}\~ao de forma interdisciplinar para promover avan\c{c}os no
processamento autom\'atico da l\'{\i}ngua portuguesa. A Lingu\'{\i}stica
Descritiva, em especial, tem enorme potencial para aportar
conhecimentos ao Processamento Autom\'atico de L\'{\i}ngua Natural (PLN), de
maneira a colocar a l\'{\i}ngua portuguesa numa posi\c{c}\~ao de destaque no
cen\'ario mundial, fazendo frente \`a grande produ\c{c}\~ao de recursos
computacionais para outras l\'{\i}nguas (como o ingl\^es, franc\^es ou
espanhol), que vislumbraram essa interdisciplinaridade j\'a na d\'ecada de
1960.

Os trabalhos aqui apresentados vinculam-se aos grandes temas da
descri\c{c}\~ao lingu\'{\i}stica do portugu\^es, a saber: Estudos de Fon\'etica e
Fonologia, Estudos do L\'exico (Lexicologia, Lexicografia e
Terminologia), Estudos de Sintaxe, Estudos de Sem\^antica, Estudos de
Texto e Discurso, nas mais diversas correntes te\'oricas. Os trabalhos
selecionados s\~ao apresentados em formato de comunica\c{c}\~ao oral ou de
p\^oster, segundo a orienta\c{c}\~ao do nosso Comit\^e Cient\'{\i}fico.  Esperamos
que os trabalhos aqui reunidos inspirem novas participa\c{c}\~oes no nosso
evento.

Nesta edi\c{c}\~ao da JDP, temos os seguintes trabalhos, apresentados por
colegas de diversas regi\~oes do Brasil e de Portugal.

%%%%%%%%%%%%%%%%%%%%%%%%%%%%%%%%%%%%%%%%%%%%%%%%%%%%%%%%%%%%%%%%%%%%%%%%%%%%%%%%%%%%%%%%%%%%%%%%%%%%%%%%%%%%%%%%%%%%%

%Chairs:

\section*{Organization Committee}
Oto Ara\'ujo Vale (UFSCar, S\~ao Carlos, SP, Brasil)\\
Ta\'{\i}sa Peres de Oliveira (UFMS, Tr\^es Lagoas, MS, Brasil)\\
Guilherme Fromm (UFU, Uberl\^andia, MG, Brasil)\\
Talita de C\'assia Marine (UFU, Uberl\^andia, MG, Brasil)


\section*{Scientific Committee}
Amanda Pontes Rassi (Lionbridge-Brasil)\\
Ang\'elica Rodrigues (Universidade Estadual Paulista, Araraquara, SP, Brasil)\\
Ariani Di Felippo (Universidade Federal de S\~ao Carlos, S\~ao Carlos, SP, Brasil)\\
Claudia Dias de Barros (Instituto Federal de S\~ao Paulo,  Sert\~aozinho, SP, Brasil)\\
Edson Rosa Francisco de Souza (Universidade Estadual Paulista, S\~ao Jos\'e do Rio Preto, SP, Brasil)\\
\'Eric Laporte (Universit\'e Paris Est, Marne-La-Vall\'ee, Fran\c{c}a)\\
Flavia Bezerra de Menezes Hirata-Vale (Universidade Federal de S\~ao Carlos, S\~ao Carlos, SP, Brasil)\\
Gladis Maria de Barcellos Almeida (Universidade Federal de S\~ao Carlos, S\~ao Carlos, SP, Brasil)\\
Guilherme Fromm (Universidade Federal de Uberl\^andia, Uberl\^andia, MG, Brasil)\\
Jorge Baptista (Universidade do Algarve, Faro, Portugal)\\
Juliano Desiderato Antonio (Universidade Estadual de Maring\'a,  Maring\'a, PR, Brasil)\\
Leonel Figueiredo de Alencar (Universidade Federal do Cear\'a, Fortaleza, CE, Brasil)\\
Marcelo M\'odolo (Universidade de S\~ao Paulo, S\~ao Paulo, SP, Brasil)\\
Margarita Correia (Universidade de Lisboa, Lisboa, Portugal)\\
Maria Jos\'e Bocorny Finatto (Universidade Federal do Rio Grande do Sul, Porto Alegre, RS, Brasil)\\
Marize Mattos Dall'Aglio-Hattnher (Universidade Estadual Paulista, S\~ao Jos\'e do Rio Preto, SP, Brasil)\\
Oto Ara\'ujo Vale (Universidade Federal de S\~ao Carlos, S\~ao Carlos, SP, Brasil)\\
Pablo Arantes  (Universidade Federal de S\~ao Carlos, S\~ao Carlos, SP, Brasil)\\
Renato Basso  (Universidade Federal de S\~ao Carlos, S\~ao Carlos, SP, Brasil)\\
Ta\'{\i}sa Peres de Oliveira (Universidade Federal de Mato Grosso do Ssul, Tr\^es Lagoas, MS, Brasil)\\
Talita de C\'assia Marine (Universidade Federal de Uberl\^andia, Uberl\^andia,  MG, Brasil)\\
Tiago Torrent (Universidade Federal de Juiz de Fora, Juiz de Fora, MG, Brasil)

\emptypage

%%%%%%%%%%%%%%%%%%%%%%%%%%%%%%%%%%%%%%%%%%%%%%%%%%%%%%%%%%%%%%%%%%%%%%%%%%%%%%%%%%%%%%%%%%%%%%%%%%%%%%%%%%%%%%%%%%%%%

%Oral presentations:

\chapter{Papers}

\emptypage

\ppjdp{180001}{8}{A Implementa\c{c}\~ao de uma Minigram\'atica do Portugu\^es Brasileiro sob a Perspectiva da LFG}{Daniel Soares and Francisco Nogueira and Leonel Figueiredo Alencar}
\ppjdp{180002}{8}{Constitui\c{c}\~ao de Um Dicion\'ario Eletr\^onico Tril\'ingue Fundado em Frames a partir da Extra\c{c}\~ao Autom\'atica de Candidatos a Termos do Dom\'inio do Turismo}{Simone Rodrigues Peron-Corr\^ea and Tiago Timponi Torrent}
\ppjdp{180003}{8}{A Modelagem Computacional do Dom\'inio dos Esportes na FrameNet Brasil}{Alexandre Diniz Costa and Tiago Timponi Torrent}
\ppjdp{180004}{8}{Descri\c{c}\~ao e modelagem de constru\c{c}\~oes interrogativas QU- em Portugu\^es Brasileiro para o desenvolvimento de um chatbot}{Nat\'alia Duarte Mar\c{c}\~ao and Tiago Timponi Torrent and Ely Edison Silva Matos}
\ppjdp{180005}{7}{Constru\c{c}\~oes de Estrutura Argumental no \^ambito do Constructicon da FrameNet Brasil: proposta de uma modelagem lingu\'istico-computacional}{V\^ania Gomes Almeida and Tiago Timponi Torrent}
\ppjdp{180006}{8}{Investiga\c{c}\~ao Preliminar Sobre a Pros\'odia Sem\^antica de Verbos de Elocu\c{c}\~ao: o Caso do Verbo "Confessar"}{Barbara C. Ramos}
\ppjdp{180007}{8}{Uma Proposta Metodol\'ogica para a Categoriza\c{c}\~ao Automatizada de Atra\c{c}\~oes Tur\'isticas a partir de Coment\'arios de Usu\'arios em Plataformas Online}{Vanessa Maria Ramos Lopes Paiva and Tiago Timponi Torrent}
\ppjdp{180008}{7}{Sofrer uma ofensa, Receber uma advert\^encia: Verbos-suporte Conversos de 'Fazer' no Portugu\^es do Brasil}{Cla\'udia D. Barros and Nathalia P. Calcia and Oto A. Vale}
\ppjdp{180009}{9}{Os Prov\'erbios em manuais de ensino de Portugu\^es L\'ingua N\~ao Materna}{S\'onia Reis and Jorge Baptista}
\ppjdp{180010}{10}{Prosody, syntax, and pragmatics: insubordination in spoken Brazilian Portuguese}{Giulia Bossaglia and Heliana Mello and Tommaso Raso}
\ppjdp{180011}{8}{As bases de dados verbais ADESSE e ViPEr: uma an\'alise constrastiva das constru\c{c}\~oes locativas em espanhol e em portugu\^es}{Roana Rodrigues and Oto Vale and Laura Alonso Alemany}
